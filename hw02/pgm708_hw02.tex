\documentclass[12pt]{article}
\usepackage{amsmath,amsthm,amssymb,MnSymbol,enumerate,fullpage}
\setlength{\parindent}{0in}

\title{PGM 10-708: HW 2}
\author{Willie Neiswanger\\
\texttt{willie@cs.cmu.edu}}
\date{}

\begin{document}

\maketitle


\section*{Problem 1}
\label{sec:prob1}

\subsection*{1.1}
Let $\textbf{x} = (x_1,\ldots,x_p) \sim p(\textbf{x} | 0, \Sigma) = \text{Normal}(0,\Sigma)$, and $\Omega = \Sigma^{-1}$.

\subsubsection*{1.1.a}
Let $\textbf{x} = (\textbf{x}_1,\textbf{x}_2)$, where $\textbf{x}_1$ and $\textbf{x}_2$ form a partition of the $p$ variables, and 
$(\textbf{x}_1,\textbf{x}_2) \sim 
\text{Normal} \left( \left[ \begin{smallmatrix} \textbf{x}_1 \\ \textbf{x}_2 \end{smallmatrix} \right] | 0, \left[ \begin{smallmatrix} \Sigma_{1,1}& \Sigma_{1,2} \\ \Sigma_{2,1} & \Sigma_{2,2} \end{smallmatrix} \right] \right)$.
Derive $p(\textbf{x}_1 | \textbf{x}_2)$.\\

Denote the block precision matrix as 
$ \left[ \begin{smallmatrix} \Omega_{1,1}& \Omega_{1,2} \\ \Omega_{2,1} & \Omega_{2,2} \end{smallmatrix} \right] = 
\left[ \begin{smallmatrix} \Sigma_{1,1}& \Sigma_{1,2} \\ \Sigma_{2,1} & \Sigma_{2,2} \end{smallmatrix} \right]^{-1} $.
Consider the quadratic form of the exponent of 
$\text{Normal} \left( \left[ \begin{smallmatrix} \textbf{x}_1 \\ \textbf{x}_2 \end{smallmatrix} \right] | 0, \left[ \begin{smallmatrix} \Sigma_{1,1}& \Sigma_{1,2} \\ \Sigma_{2,1} & \Sigma_{2,2} \end{smallmatrix} \right] \right)$
i.e., $-\frac{1}{2} \textbf{x}^{\top} \Sigma^{-1} \textbf{x} = 
-\frac{1}{2} \textbf{x}_1^{\top} \Omega_{11} \textbf{x}_1 
-\frac{1}{2} \textbf{x}_1^{\top} \Omega_{12} \textbf{x}_2   
-\frac{1}{2} \textbf{x}_2^{\top} \Omega_{21} \textbf{x}_1 
-\frac{1}{2} \textbf{x}_2^{\top} \Omega_{22} \textbf{x}_2
$. As this is still a quadratic form, we know that $p(\textbf{x}_1 | \textbf{x}_2)$ is still a Gaussian. The goal is now to find the mean and covariance of this distribution.
%Note that the quadratic form of the original exponent can be written
%$-\frac{1}{2} \textbf{x}^{\top} \Sigma^{-1} \textbf{x} = 
By inspecting the above equality, we can pick out the first term 
$-\frac{1}{2} \textbf{x}_1^{\top} \Omega_{11} \textbf{x}_1$. Regarding $\textbf{x}_1$ as our observation and $\textbf{x}_2$ as a constant, this has the same form of our original Gaussian, 
and implies that the covariance for the conditional distribution 
\begin{equation}
    \Sigma_{1|2} = \Omega_{11}^{-1}.
\end{equation}
Through inspection we see that the terms linear in $\textbf{x}_1$ are 
$\textbf{x}_1^{\top}[\Omega_{12}\textbf{x}_2]$ (where we've used the fact that $\Omega_{21}^{\top} = \Omega_{12}$).
This implies that the coefficient of $\textbf{x}_1$ in this expression must equal $\Sigma_{1|2}^{-1} \mu_{1|2}$, and hence
\begin{equation}
    \mu_{1|2} = \Sigma_{1|2}[\Omega_{12}\textbf{x}_2] = - \Omega_{11}^{-1} \Omega_{12} \textbf{x}_2
\end{equation}
In order to write $\Sigma_{1|2}$ and $\mu_{1|2}$ in terms of the original parameter $\Sigma$, we use the Schur complement of the matrix, using the definition
$\left[ \begin{smallmatrix} \Sigma_{1,1}& \Sigma_{1,2} \\ \Sigma_{2,1} & \Sigma_{2,2} \end{smallmatrix} \right]^{-1} = 
 \left[ \begin{smallmatrix} \Omega_{1,1}& \Omega_{1,2} \\ \Omega_{2,1} & \Omega_{2,2} \end{smallmatrix} \right]$,
 we have the relation
 \begin{equation}
    \begin{split}
        \Omega_{11} &= (\Sigma_{11} - \Sigma_{12} \Sigma_{22}^{-1} \Sigma_{21})\\
        \Omega_{12} &= - (\Sigma_{11} - \Sigma_{12} \Sigma_{22}^{-1} \Sigma_{21})^{-1} \Sigma_{12} \Sigma_{22}^{-1}
    \end{split}
\end{equation}
We substitute these in the expressions above for $\Sigma_{1|2}$ and $\mu_{1|2}$ to get
\begin{equation}
    \begin{split}
        \mu_{1|2} &= \Sigma_{12} \Sigma_{22}^{-1} \textbf{x}_2 \\
        \Sigma_{1|2} &=  \Sigma_{11} - \Sigma_{12} \Sigma_{22}^{-1} \Sigma_{21}
    \end{split}
\end{equation}


\subsubsection*{1.1.b}
We can write the precision matrix in block form as $\Omega = \left[ 
\begin{smallmatrix} \Omega_{1,1} & \Omega_{1,2} \\ \Omega_{2,1} & \Omega_{2,2}
\end{smallmatrix} \right]$.
Derive $\text{Var}(\textbf{x}_1 | \textbf{x}_2)$ in terms of $\Omega$.\\

Using part of the derivation from (1.1.a), we know that $\textbf{x}_1 | \textbf{x}_2$ is normally distributed, and that the variance in terms of $\Omega$ can be written
\begin{equation}
    \Sigma_{1|2} = \Omega_{11}^{-1}.
\end{equation}


\subsubsection*{1.1.c}
Argue that $\Omega_{i,j}=0$ iff $x_i \upmodels x_j | \text{rest}$.\\

Let $\textbf{x}_1 = (x_i,x_j)$ and $\textbf{x}_2 = \textbf{x}_{-(i,j)}$ (i.e. $\textbf{x}_2$ is all other $x_i$'s). Then, if $\Omega_{i,j} = 0$, 
\\

If $x_i \upmodels x_j | \text{rest}$ $\implies$ $p(\textbf{x}_1 | \textbf{x}_2) = p(x_i | \textbf{x}_2) p(x_j | \textbf{x}_2)$


\subsection*{1.2}
Let $\textbf{x}_1, \ldots, \textbf{x}_n$ $\sim$ $\text{Normal}(0,\Sigma)$, where $\textbf{x}_i = (x_i^1, \ldots x_i^p)$. Let $\Theta = \Sigma^{-1}$ and $S = \sum_i \frac{\textbf{x}_i \textbf{x}_i^\top}{n}$. Show that the log-likelihood of the multivariate Normal distribution to to be optimized is equal to $\text{log} \text{ det} \Theta - \text{tr}(S\Theta)$.
\\

$\text{log}(\text{Normal}(0,\Theta^{-1}))$ $=$ 
$-\frac{p}{2}\text{log}(2\pi) - \frac{1}{2} \text{log} \text{ det}(\Theta^{-1}) - \frac{1}{2}\textbf{x}^{\top} \Theta \textbf{x}$ 
$=$ 
$-\frac{p}{2}\text{log}(2\pi) + \frac{1}{2} \text{log} \text{ det}(\Theta) - \frac{1}{2}$
Optimizing this is therefore equivalent to optimizing $\text{log} \text{ det}(\Theta) - \text{tr}(S\Theta)$.


\subsection*{1.3}
Implement the Meinshausen-Buhlmann and Glasso algorithms.

The estimated precision matrices are drawn below. It seems that the estimated precision matrices become more sparse as the lambda parameter is increased.



\section*{Problem 2}
\label{sec:prob2}
Assume, for simplicity, that we denote $P(z_1) = P(z_1 | z_0) = \eta_{z_0,z_1}$

\subsection*{2.1}
The parameters of this model are $\theta$ (dimension $K$), $\sigma^2$ (dimension $K$), $\eta$ (a matrix of size $K \times K$), and $\gamma$ (dimension $K$).

\subsection*{2.2}
%Write the expected conditional log-likelihood for data $(x_1,y_1),\ldots,(x_n,y_n)$.
%\\

The conditional log-likelihood can be written
\begin{equation*}
    \begin{split}
        &\text{log}(P(\textbf{Y} | \textbf{X})) = \text{log} \left( \sum_{\textbf{Z}} P(\textbf{Y},\textbf{Z} | \textbf{X}) \right)
             = \text{log} \left( \sum_{\textbf{Z}} P(\textbf{Z} | \textbf{X}) P(\textbf{Y} | \textbf{Z}, \textbf{X}) \right)\\
        & =  \text{log} \left( \mathbb{E}_q \left[ P(\textbf{Z} | \textbf{X}) P(\textbf{Y} | \textbf{Z}, \textbf{X}) \frac{1}{q(\textbf{Z})} \right] \right)
             \text{ \hspace{3mm}(for a distribution $q$ over $Z$)}\\
        & \geq  \mathbb{E}_q \left[  \text{log} \left( P(\textbf{Z} | \textbf{X}) P(\textbf{Y} | \textbf{Z}, \textbf{X}) 
            \frac{1}{q(\textbf{Z})} \right) \right]
             \text{ \hspace{3mm}(using Jensen's inequality)}\\
        & = \mathbb{E}_q \left[ \text{log} \left( P(\textbf{Z} | \textbf{X}) \right) \right] 
             + \mathbb{E}_q \left[\text{log} \left( P(\textbf{Y} | \textbf{Z}, \textbf{X}) \right) \right] 
             - \mathbb{E}_q \left[\text{log} \left( q(\textbf{Z}) \right) \right] \\
        & = \mathbb{E}_q \left[ \sum_i \text{log} \left( P(z_i | z_{i-1}, x_i) \right)
             + \sum_i \text{log} \left( P(y_i | z_i, x_i) \right) \right] 
             - \mathbb{E}_q \left[\text{log} \left( q(\textbf{Z}) \right) \right] \\
        & = \mathbb{E}_q \left[ \sum_i (\text{log}(\eta_{z_{i-1},z_i}) + \gamma_{z_i}^{\top} x_i )
             + \sum_i \text{log} \left( \text{Normal}(y_i | \theta_{z_i}^{\top} x_i, \sigma_{z_i}^2) \right) \right] 
             - \mathbb{E}_q \left[\text{log} \left( q(\textbf{Z}) \right) \right] \\
        & = \mathbb{E}_q \left[ \sum_i \left( \text{log}(\eta_{z_{i-1},z_i}) + \gamma_{z_i}^{\top} x_i 
             -\frac{1}{2}\left( \text{log}(2\pi) - \text{log}(\sigma_{z_i}^2) - \frac{(y_i | \theta_{z_i}^{\top} x_i)^2}{ \sigma_{z_i}^2} \right)  \right) \right]
             - \mathbb{E}_q \left[\text{log} \left( q(\textbf{Z}) \right) \right] \\
         & =  \sum_i \sum_{z_i} \sum_{z_{i-1}} \left( \text{log}(\eta_{z_{i-1},z_i}) + \gamma_{z_i}^{\top} x_i 
             -\frac{1}{2}\left( \text{log}(2\pi) - \text{log}(\sigma_{z_i}^2) - \frac{(y_i | \theta_{z_i}^{\top} x_i)^2}{ \sigma_{z_i}^2} \right)  \right)
             - \mathbb{E}_q \left[\text{log} \left( q(\textbf{Z}) \right) \right] \\
        %
        %
        %& = \sum_i \left( \mathbb{E}_q\left[ \text{log}(\eta_{z_{i-1},z_i}) \right] + \mathbb{E}_q\left[ \gamma_{z_i}^{\top} x_i \right]
             %-\frac{1}{2}\left( \text{log}(2\pi) - \mathbb{E}_q\left[ \text{log}(\sigma_{z_i}^2) \right] - 
                %\mathbb{E}_q\left[ \frac{(y_i | \theta_{z_i}^{\top} x_i)^2}{ \sigma_{z_i}^2} \right] \right)  \right)
             %- \mathbb{E}_q \left[\text{log} \left( q(\textbf{Z}) \right) \right] \\
        %
        %
        %& = \text{log} \left( \prod_{i=1}^p \sum_{z_i}  e^{\gamma_{z_i}^{\top} x_i} \eta_{z_{i-1},z_i} 
            %\text{Normal}(y_i | \theta_{z_i}^{\top} x_i, \sigma_{z_i}^2) \right)\\
        %& = \sum_{i=1}^p \text{log} \left( \sum_z e^{\gamma_{z_i}^{\top} x_i} \eta_{z_{i-1},z_i} 
            %\text{Normal}(y_i | \theta_{z_i}^{\top} x_i, \sigma_{z_i}^2) \right)\\
        %&\sum_{i=2}^p \left( \gamma_{z_i}^{\top} x_i + \text{log}(\eta_{z_{i-1},z_i}) + 
            %\text{log}(\text{Normal}(y_i | \theta_{z_i}^{\top} x_i, \sigma_{z_i}^2)) \right)
    \end{split}
\end{equation*}
This is the EM objective function; note that the second element of the final term is the entropy.\\
%Therefore, the expected conditional log-likelihood is 
%\begin{equation}
    %\begin{split}
        %&\mathbb{E} \left[ \text{log} (P(\textbf{Y} | \textbf{X})) \right]
            %= \sum_{i=1}^p \mathbb{E} \left[ \text{log} \left( \sum_z e^{\gamma_{z_i}^{\top} x_i} \eta_{z_{i-1},z_i} 
            %\text{Normal}(y_i | \theta_{z_i}^{\top} x_i, \sigma_{z_i}^2) \right) \right] \\
        %&\mathbb{E} \left[ \text{log} (P(\textbf{Y} | \textbf{X})) \right]
    %\end{split}
%\end{equation}
%Using Jensen's inequality we can provide a lower bound to the above quantity with 
%\begin{equation}
    %\begin{split}
        %&\mathbb{E} \left[ \text{log} (P(\textbf{Y} | \textbf{X})) \right] \leq 2\\
        %&\mathbb{E} \left[ \text{log} (P(\textbf{Y} | \textbf{X})) \right] \leq 2
    %\end{split}
%\end{equation}

\subsection*{2.3}
Derive the update equations for the EM algorithm for this model.
\\

Let the EM object function above be written $\mathcal{L}(\theta,q)$, where $\theta$ are the model parameters and $q$ is the distribution over $\textbf{Z}$. For a version of EM, we can formulate the E-step as updating the distribution $q$ in the following way: 
\begin{equation}
    \begin{split}
        & q^{(t+1)} = \text{argmax}_q \mathcal{L}(q,\theta^{(t)}) = P(\textbf{Z} | \textbf{Y}, \textbf{X}, \theta^{(t)})
            = \frac{1}{K} P(\textbf{Y} | \textbf{Z}, \textbf{X}, \theta^{(t)}) P(\textbf{Z} | \textbf{X}, \theta^{(t)}) \\
        & = \frac{1}{K} \prod_i P(y_i | x_i, z_i) \prod_i P(z_i | x_i, z_{i-1})
            = \frac{1}{K} \prod_i \text{Normal}(y_i | \theta_{z_i}^{\top} x_i, \sigma_{z_i}^2)  
            \eta_{z_{i-1},z_i} e^{\gamma_{z_i}^{\top} x_i}
    \end{split}
\end{equation}
where in this case the normalizing constant $K$ is computed by summing over the latent variables $\textbf{Z}$ in the numerator.
\\

And we can write the M-step as:
\begin{equation}
    \theta^{(t)} = \text{argmax}_{\theta} \mathcal{L}(q^{(t+1)},\theta)
\end{equation}
This involves maximizing the objective function 
\begin{equation}
    \begin{split}
        & 2\sum_i \sum_{z_i} \sum_{z_{i-1}} \left( \text{log}(\eta_{z_{i-1},z_i}) + \gamma_{z_i}^{\top} x_i 
             -\frac{1}{2}\left( \text{log}(2\pi) - \text{log}(\sigma_{z_i}^2) - \frac{(y_i | \theta_{z_i}^{\top} x_i)^2}{ \sigma_{z_i}^2} \right)  \right)
    \end{split}
\end{equation}
with respect to the parameters $\eta_{z_{i-1},z_i}$, $\gamma_{z_i}$, $\theta_{z_i}$, and $\sigma_{z_i}^2$.


\section*{Problem 3}
\label{sec:prob3}
The KL divergence can be written 
\begin{equation}
    \begin{split}
        &\text{KL}(P(x|\theta),P(x|\eta)) = \int_x \log\left(\frac{P(x|\theta)}{P(x|\eta)}\right)P(x|\theta) 
            = \int_x \left( \text{log}P(x|\theta) - \text{log}P(x|\eta) \right) P(x|\theta) \\
        & = \int_x P(x|\theta) \text{log}P(x|\theta) - \int_x P(x|\theta) \text{log}P(x|\eta) \\
        & = \int_x P(x|\theta) \left( \sum_{i=1}^k \theta_i f_i(x) - A(\theta)  \right) -   
                \int_x P(x|\theta) \left( \sum_{i=1}^k \eta_i f_i(x) - A(\eta)  \right) \\
        & = \int_x P(x|\theta) \sum_{i=1}^k \theta_i f_i(x) - \int_x P(x|\theta) A(\theta)  -   
                \int_x P(x|\theta) \sum_{i=1}^k \eta_i f_i(x) - \int_x P(x|\theta) A(\eta) \\
        & = \int_x P(x|\theta) \sum_{i=1}^k \theta_i f_i(x) - \int_x P(x|\theta) \sum_{i=1}^k \eta_i f_i(x) 
            - A(\theta) + A(\eta)\\
        & = \int_x P(x|\theta) \left( \sum_{i=1}^k (\theta_i - \eta_i) f_i(x) \right)- A(\theta) + A(\eta)\\
        & = \sum_{i=1}^k (\theta_i - \eta_i) \int_x P(x|\theta) f_i(x) - A(\theta) + A(\eta)\\
        & = \sum_{i=1}^K (\theta_i - \eta_i) \frac{\partial A(\theta)}{\partial \theta_i} - A(\theta) + A(\eta)
    \end{split}
\end{equation}
To show the last equality holds, we need to show that $\int_x P(x|\theta) f_i(x) = \frac{\partial A(\theta)}{\partial \theta_i}$. This holds because $P(X|\theta) = \frac{\text{exp}(\sum_i \theta_i f_i(x))}{\text{exp}(A(\theta))}$, hence due to normalization $A(\theta) = \text{log}\int_x \text{exp}(\sum_i \theta_i f_i(x))$.
Therefore,
\begin{equation}
    \begin{split}
        \frac{\partial A(\theta)}{\partial \theta_j}
            &= \frac{1}{\int_x \text{exp}(\sum_i \theta_i f_i(x))}   \int_x \text{exp}(\sum_i \theta_i f_i(x)) f_j(x) \\
            &= \frac{1}{\text{exp}(A(\theta))} \int_x \text{exp}(\sum_i \theta_i f_i(x)) f_j(x) 
                = \int_x \text{exp}(\sum_i \theta_i f_i(x) - A(\theta)) f_j(x) \\
            &= \int_x P(X|\theta)f_j(x)
    \end{split}
\end{equation}
Hence $\int_x P(x|\theta) f_i(x) = \frac{\partial A(\theta)}{\partial \theta_i}$ and this completes the proof.


\section*{Problem 4}
\label{sec:prob4}

The IPF update rule is: $\Psi_C(x_C)^{(t+1)} = \Psi_C(x_C)^{(t)} \frac{\tilde{p}(x_C)}{p^{(t)}(x_C)}$


\end{document}
